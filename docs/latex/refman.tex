\documentclass[11pt]{book}
\usepackage[english]{babel}
\usepackage{hyperref}
\usepackage{geometry}
\usepackage{listings}
\usepackage{xcolor}
\usepackage{tabularx}
\usepackage{multirow}
\usepackage{framed}
\usepackage{float}

\geometry{a4paper,left=2.5cm,right=2.5cm,top=2.5cm,bottom=2.5cm}

\lstset{
    language=C++,
    basicstyle=\ttfamily\footnotesize,
    keywordstyle=\color{blue}\bfseries,
    commentstyle=\color{gray}\itshape,
    stringstyle=\color{red},
    showstringspaces=false,
    breaklines=true,
    frame=single,
    numbers=left,
    numberstyle=\tiny\color{gray},
    captionpos=b,
    tabsize=2
}

\title{VEALC Server \\ Complete Project Documentation}
\author{Generated from Source Code}
\date{Version 1.0 \\ \today}

\begin{document}
\maketitle
\tableofcontents

\chapter*{Introduction}
\addcontentsline{toc}{chapter}{Introduction}

VEALC (Vector Encryption Authentication and Logic Calculator) Server is a TCP server application written in C++ that provides:

\begin{itemize}
\item Client authentication using MD5 hashing with salt
\item Vector data processing with overflow control
\item Concurrent client session management
\item Comprehensive logging system
\item Configurable server settings
\end{itemize}

\chapter{Project Architecture}

\section{System Overview}
The server follows a modular architecture with clear separation of concerns:

\begin{figure}[H]
\centering
\begin{framed}
\begin{verbatim}
        +---------------+
        |    Client     |
        +-------+-------+
                |
        +-------+-------+
        |    Session    |
        +-------+-------+
                |
    +-----------+-----------+
    |                       |
+---+-------+         +-----+-----+
|  Auth     |         |  Vector   |
|  Module   |         |  Processor|
+-----------+         +-----------+
    |                       |
+---+-------+         +-----+-----+
|  Logger   |         |   Config  |
+-----------+         +-----------+
\end{verbatim}
\end{framed}
\caption{VEALC Server Architecture}
\end{figure}

\section{Component Responsibilities}
\begin{tabularx}{\textwidth}{|l|X|}
\hline
\textbf{Component} & \textbf{Responsibility} \\
\hline
\texttt{Server} & Main server class, manages connections and sessions \\
\hline
\texttt{Session} & Handles individual client communication \\
\hline
\texttt{Authenticator} & Client authentication using MD5 \\
\hline
\texttt{VectorProcessor} & Mathematical vector calculations \\
\hline
\texttt{Logger} & Logging system for events and errors \\
\hline
\texttt{Config} & Server configuration management \\
\hline
\end{tabularx}

\chapter{Class Documentation}

\section{Class: Server}
\label{class:server}

\subsection{Class Definition}
\begin{lstlisting}[caption=Server Class Header]
class Server {
private:
    ServerConfig config_;
    Logger logger_;
    std::unordered_map<std::string, std::string> clients_;
    int server_fd_;
    
    void load_clients();
    void setup_socket();
    void accept_connections();
    
public:
    Server(const ServerConfig& config);
    ~Server();
    void run();
};
\end{lstlisting}

\subsection{Public Methods}
\begin{tabularx}{\textwidth}{|l|X|}
\hline
\textbf{Method} & \textbf{Description} \\
\hline
\texttt{Server(const ServerConfig\& config)} & Constructor with configuration \\
\hline
\texttt{\~Server()} & Destructor, closes socket \\
\hline
\texttt{void run()} & Starts the main server loop \\
\hline
\end{tabularx}

\subsection{Private Methods}
\begin{tabularx}{\textwidth}{|l|X|}
\hline
\textbf{Method} & \textbf{Description} \\
\hline
\texttt{void load\_clients()} & Loads client database from file \\
\hline
\texttt{void setup\_socket()} & Configures server socket \\
\hline
\texttt{void accept\_connections()} & Accepts incoming client connections \\
\hline
\end{tabularx}

\section{Class: Session}
\label{class:session}

\subsection{Class Definition}
\begin{lstlisting}[caption=Session Class Header]
class Session {
private:
    int client_socket;
    std::unordered_map<std::string, std::string>& clients;
    Logger& logger;
    std::string receive_buffer;
    
    void receive_to_buffer();
    std::string extract_from_buffer_until_non_hex();
    std::string extract_from_buffer_exact(size_t length);
    bool verify_authentication(const std::string& login, 
                              const std::string& salt, 
                              const std::string& received_hash);
    void process_vectors();
    uint32_t receive_uint32();
    std::vector<int32_t> receive_vector(uint32_t size);
    void send_uint32(uint32_t value);
    void send_int32(int32_t value);
    std::string calculate_md5(const std::string& data);
    int32_t calculate_vector_product(const std::vector<int32_t>& vector);
    
public:
    Session(int client_socket, 
            std::unordered_map<std::string, std::string>& clients, 
            Logger& logger);
    void handle();
    bool send_text(const std::string& text);
};
\end{lstlisting}

\subsection{Key Methods}
\begin{itemize}
\item \texttt{handle()} - Main session handler
\item \texttt{verify\_authentication()} - Client authentication
\item \texttt{calculate\_vector\_product()} - Vector multiplication with overflow control
\end{itemize}

\section{Class: Authenticator}
\label{class:authenticator}

\subsection{Class Definition}
\begin{lstlisting}[caption=Authenticator Class Header]
class Authenticator {
public:
    static std::string generate_salt_16();
    static std::string calculate_md5_hash(const std::string& salt, 
                                          const std::string& password);
    static bool verify_client(const std::string& login, 
                              const std::string& received_hash, 
                              const std::string& salt, 
                              const std::unordered_map<std::string, std::string>& clients);
};
\end{lstlisting}

\subsection{Static Methods}
\begin{itemize}
\item \texttt{generate\_salt\_16()} - Generates 16-byte random salt
\item \texttt{calculate\_md5\_hash()} - Computes MD5(salt + password)
\item \texttt{verify\_client()} - Verifies client credentials
\end{itemize}

\section{Class: VectorProcessor}
\label{class:vectorprocessor}

\subsection{Class Definition}
\begin{lstlisting}[caption=VectorProcessor Class Header]
class VectorProcessor {
public:
    static int32_t calculate_product(const Vector& vector);
    static std::vector<int32_t> multiply_vectors(const std::vector<Vector>& vectors);
};
\end{lstlisting}

\subsection{Algorithm Implementation}
\begin{lstlisting}[caption=Vector Product Calculation with Overflow Control]
int32_t VectorProcessor::calculate_product(const Vector& vector) {
    if (vector.empty()) return 0;
    
    int64_t product = 1;
    for (int32_t val : vector) {
        int64_t val64 = static_cast<int64_t>(val);
        
        // Overflow check
        if (val64 != 0 && llabs(product) > INT64_MAX / llabs(val64)) {
            return (product > 0 && val64 > 0) ? INT32_MAX : INT32_MIN;
        }
        product *= val64;
    }
    
    // Clamp to int32 range
    if (product > INT32_MAX) return INT32_MAX;
    if (product < INT32_MIN) return INT32_MIN;
    return static_cast<int32_t>(product);
}
\end{lstlisting}

\section{Class: Logger}
\label{class:logger}

\subsection{Class Definition}
\begin{lstlisting}[caption=Logger Class Header]
class Logger {
private:
    std::string log_file_;
    std::string get_current_time();
    
public:
    Logger(const std::string& filename);
    void log(const std::string& message, bool critical = false);
    void log_add(const std::string& message);
    void log_error(const std::string& error, bool critical = false);
};
\end{lstlisting}

\subsection{Log Format}
\begin{verbatim}
[2024-01-15 10:30:00] [CRITICAL] err: Error message
[2024-01-15 10:30:05] [NON-CRITICAL] Info message
\end{verbatim}

\section{Struct: ServerConfig}
\label{struct:serverconfig}

\subsection{Structure Definition}
\begin{lstlisting}[caption=ServerConfig Structure]
struct ServerConfig {
    std::string client_db_file = "/etc/vealc.conf";
    std::string log_file = "/var/log/vealc.log";
    int port = 33333;
    
    static ServerConfig parse_args(int argc, char* argv[]);
    static void print_help();
};
\end{lstlisting}

\chapter{File Documentation}

\section{Header Files (.h)}

\subsection{auth.h}
\begin{itemize}
\item \textbf{Purpose:} Client authentication declarations
\item \textbf{Contains:} Authenticator class
\item \textbf{Dependencies:} \texttt{<string>}, \texttt{<unordered\_map>}
\end{itemize}

\subsection{config.h}
\begin{itemize}
\item \textbf{Purpose:} Server configuration
\item \textbf{Contains:} ServerConfig structure
\item \textbf{Dependencies:} \texttt{<string>}
\end{itemize}

\subsection{logger.h}
\begin{itemize}
\item \textbf{Purpose:} Logging system interface
\item \textbf{Contains:} Logger class
\item \textbf{Dependencies:} \texttt{<string>}
\end{itemize}

\subsection{session.h}
\begin{itemize}
\item \textbf{Purpose:} Client session management
\item \textbf{Contains:} Session class
\item \textbf{Dependencies:} \texttt{<string>}, \texttt{<vector>}, \texttt{<unordered\_map>}
\end{itemize}

\subsection{server.h}
\begin{itemize}
\item \textbf{Purpose:} Main server interface
\item \textbf{Contains:} Server class
\item \textbf{Dependencies:} \texttt{"config.h"}, \texttt{"logger.h"}, \texttt{<unordered\_map>}
\end{itemize}

\subsection{types.h}
\begin{itemize}
\item \textbf{Purpose:} Common type definitions
\item \textbf{Contains:} Vector, ByteArray type aliases
\item \textbf{Dependencies:} \texttt{<cstdint>}, \texttt{<vector>}
\end{itemize}

\subsection{vector\_processor.h}
\begin{itemize}
\item \textbf{Purpose:} Vector calculations interface
\item \textbf{Contains:} VectorProcessor class
\item \textbf{Dependencies:} \texttt{"types.h"}, \texttt{<cstdint>}, \texttt{<vector>}
\end{itemize}

\section{Source Files (.cpp)}

\subsection{auth.cpp}
\begin{itemize}
\item \textbf{Purpose:} Authentication implementation
\item \textbf{Implements:} Authenticator methods
\item \textbf{Dependencies:} \texttt{"auth.h"}, \texttt{<openssl/md5.h>}, \texttt{<random>}
\end{itemize}

\subsection{config.cpp}
\begin{itemize}
\item \textbf{Purpose:} Configuration parsing
\item \textbf{Implements:} ServerConfig methods
\item \textbf{Dependencies:} \texttt{"config.h"}, \texttt{<iostream>}, \texttt{<cstring>}
\end{itemize}

\subsection{logger.cpp}
\begin{itemize}
\item \textbf{Purpose:} Logging implementation
\item \textbf{Implements:} Logger methods
\item \textbf{Dependencies:} \texttt{"logger.h"}, \texttt{<fstream>}, \texttt{<ctime>}
\end{itemize}

\subsection{main.cpp}
\begin{itemize}
\item \textbf{Purpose:} Program entry point
\item \textbf{Contains:} main() function
\item \textbf{Dependencies:} \texttt{"server.h"}, \texttt{"config.h"}
\end{itemize}

\subsection{session.cpp}
\begin{itemize}
\item \textbf{Purpose:} Session handling implementation
\item \textbf{Implements:} Session methods
\item \textbf{Dependencies:} \texttt{"session.h"}, \texttt{<openssl/md5.h>}, \texttt{<sys/socket.h>}
\end{itemize}

\subsection{server.cpp}
\begin{itemize}
\item \textbf{Purpose:} Server implementation
\item \textbf{Implements:} Server methods
\item \textbf{Dependencies:} \texttt{"server.h"}, \texttt{"session.h"}, \texttt{<sys/socket.h>}
\end{itemize}

\subsection{vector\_processor.cpp}
\begin{itemize}
\item \textbf{Purpose:} Vector calculations implementation
\item \textbf{Implements:} VectorProcessor methods
\item \textbf{Dependencies:} \texttt{"vector\_processor.h"}, \texttt{<climits>}
\end{itemize}

\chapter{Communication Protocol}

\section{Authentication Phase}
\begin{verbatim}
Client → Server: login[16_hex_salt][32_hex_hash]
Server → Client: "OK\n" or "err\n"
\end{verbatim}

\section{Data Processing Phase}
\begin{verbatim}
Client → Server: vector_count (uint32, 4 bytes)
For each vector (vector_count times):
    Client → Server: vector_size (uint32, 4 bytes)
    Client → Server: vector_data (int32[vector_size], 4*size bytes)
    Server → Client: product (int32, 4 bytes)
\end{verbatim}

\section{Example Session}
\begin{lstlisting}[caption=Example Protocol Exchange]
// Authentication
Client sends: "alice4F3A1B8C9D2E7F5A32B4C6D8E0F1A2B4"
Server sends: "OK\n"

// Vector processing
Client sends: 0x00000002 (2 vectors)

// First vector
Client sends: 0x00000003 (3 elements)
Client sends: 0x00000002 0x00000003 0x00000004 (2, 3, 4)
Server sends: 0x00000018 (24)

// Second vector
Client sends: 0x00000002 (2 elements)
Client sends: 0x7FFFFFFF 0x00000002 (INT32_MAX, 2)
Server sends: 0x7FFFFFFF (INT32_MAX, overflow detected)
\end{lstlisting}

\chapter{Configuration}

\section{Client Database Format}
Client credentials are stored in \texttt{/etc/vealc.conf}:
\begin{verbatim}
username:password
alice:P@ssw0rd1
bob:Secret123
charlie:Qwerty!@#
\end{verbatim}

\section{Command Line Arguments}
\begin{tabularx}{\textwidth}{|l|l|X|}
\hline
\textbf{Option} & \textbf{Default} & \textbf{Description} \\
\hline
\texttt{-p PORT} & 33333 & Server port number \\
\hline
\texttt{-c FILE} & \texttt{/etc/vealc.conf} & Client database file \\
\hline
\texttt{-d FILE} & (same as \texttt{-c}) & Alias for \texttt{-c} \\
\hline
\texttt{-l FILE} & \texttt{/var/log/vealc.log} & Log file location \\
\hline
\texttt{-h, --help} & - & Show help message \\
\hline
\end{tabularx}

\section{Example Usage}
\begin{verbatim}
# Default configuration
./server

# Custom port and config
./server -p 44444 -c ./myclients.conf -l ./server.log

# Show help
./server --help
\end{verbatim}

\chapter{Build and Deployment}

\section{Compilation}
\begin{lstlisting}[caption=Compilation Command]
g++ -std=c++11 -o server \
    main.cpp server.cpp session.cpp \
    auth.cpp config.cpp logger.cpp \
    vector_processor.cpp \
    -lssl -lcrypto
\end{lstlisting}

\section{Dependencies}
\begin{itemize}
\item \textbf{Compiler:} g++ with C++11 support
\item \textbf{Libraries:} OpenSSL (libssl-dev)
\item \textbf{System:} Linux with POSIX sockets
\end{itemize}

\section{Installation}
\begin{enumerate}
\item Install dependencies: \texttt{sudo apt-get install g++ libssl-dev}
\item Compile the server: \texttt{make} or \texttt{g++ ...}
\item Create configuration: \texttt{sudo cp vealc.conf /etc/}
\item Set up log file: \texttt{sudo touch /var/log/vealc.log}
\item Run server: \texttt{./server}
\end{enumerate}

\chapter{Security Considerations}

\section{Current Implementation}
\begin{itemize}
\item \textbf{Authentication:} MD5(salt + password)
\item \textbf{Password Storage:} Plain text in file
\item \textbf{Network:} Unencrypted TCP
\item \textbf{Salt:} 16-byte random hex string
\end{itemize}

\section{Security Limitations}
\begin{framed}
\textbf{Warning:} This implementation is for educational purposes only. 
Production use requires:
\begin{itemize}
\item Replace MD5 with SHA-256 or bcrypt
\item Encrypt network traffic with TLS/SSL
\item Hash passwords before storage
\item Implement rate limiting
\item Add input validation and sanitization
\item Use secure random number generation
\end{itemize}
\end{framed}

\chapter{Troubleshooting}

\section{Common Issues}
\begin{tabularx}{\textwidth}{|l|X|X|}
\hline
\textbf{Issue} & \textbf{Cause} & \textbf{Solution} \\
\hline
\texttt{Bind failed} & Port already in use & Use different port or kill existing process \\
\hline
\texttt{Cannot open client database} & File doesn't exist or no permissions & Check file path and permissions \\
\hline
\texttt{Authentication failed} & Wrong password or hash calculation & Verify password in config file \\
\hline
\texttt{Connection refused} & Server not running or firewall & Start server and check firewall \\
\hline
\end{tabularx}

\section{Debugging}
\begin{itemize}
\item Check logs: \texttt{tail -f /var/log/vealc.log}
\item Verify configuration: \texttt{cat /etc/vealc.conf}
\item Test connectivity: \texttt{netstat -tlnp | grep 33333}
\item Monitor processes: \texttt{ps aux | grep server}
\end{itemize}

\appendix

\chapter{Type Definitions}

\section{Vector Type}
\begin{lstlisting}
using Vector = std::vector<int32_t>;
\end{lstlisting}

\section{ByteArray Type}
\begin{lstlisting}
using ByteArray = std::vector<uint8_t>;
\end{lstlisting}

\chapter{Complete File Listing}

\section{Project Structure}
\begin{verbatim}
vealc-server/
├── include/
│   ├── auth.h
│   ├── config.h
│   ├── logger.h
│   ├── server.h
│   ├── session.h
│   ├── types.h
│   └── vector_processor.h
├── src/
│   ├── auth.cpp
│   ├── config.cpp
│   ├── logger.cpp
│   ├── main.cpp
│   ├── server.cpp
│   ├── session.cpp
│   └── vector_processor.cpp
├── Doxyfile
├── Makefile
├── vealc.conf
└── README.md
\end{verbatim}

\chapter{Index}

\section{Classes}
\begin{itemize}
\item \hyperref[class:server]{Server} - Main server class
\item \hyperref[class:session]{Session} - Client session handler
\item \hyperref[class:authenticator]{Authenticator} - Authentication module
\item \hyperref[class:vectorprocessor]{VectorProcessor} - Vector calculations
\item \hyperref[class:logger]{Logger} - Logging system
\item \hyperref[struct:serverconfig]{ServerConfig} - Configuration structure
\end{itemize}

\section{Files}
\begin{itemize}
\item \texttt{auth.h/cpp} - Authentication
\item \texttt{config.h/cpp} - Configuration
\item \texttt{logger.h/cpp} - Logging
\item \texttt{main.cpp} - Entry point
\item \texttt{session.h/cpp} - Session handling
\item \texttt{server.h/cpp} - Server implementation
\item \texttt{types.h} - Common types
\item \texttt{vector\_processor.h/cpp} - Vector processing
\end{itemize}

\chapter*{Conclusion}
\addcontentsline{toc}{chapter}{Conclusion}

VEALC Server provides a complete solution for client authentication and vector data processing. The modular architecture allows for easy maintenance and extension. While suitable for educational purposes, production deployment would require additional security measures.

\vspace{1cm}
\begin{center}
\textbf{© 2024 VEALC Server Development Team}
\end{center}

\end{document}
